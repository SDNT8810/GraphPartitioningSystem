\chapter{Implementation and System Design}

\section{Distributed System Architecture}
This section outlines the system architecture:
\begin{itemize}
    \item Overview of the distributed system components:
        \begin{itemize}
            \item Data sources
            \item Processing nodes
            \item Communication infrastructure
        \end{itemize}
    \item Integration of self-partitioning graph model
    \item System component interactions and data flow
\end{itemize}

\section{Implementation of Autonomous Partitioning Algorithms}
This section details the implementation aspects:
\begin{itemize}
    \item Implementation details of autonomous algorithms:
        \begin{itemize}
            \item Node-level decision making
            \item Local metric computation
            \item Partition management
        \end{itemize}
    \item Communication protocols and data structures
    \item Information exchange mechanisms
\end{itemize}

\section{Integration with Task-Parallel Programming}
This section explains the task-parallel integration:
\begin{itemize}
    \item Integration with task-parallel systems:
        \begin{itemize}
            \item Taskflow system utilization
            \item Task dependency management
            \item Parallel execution capabilities
        \end{itemize}
    \item Mapping of graph nodes to tasks
    \item Data stream processing within tasks
\end{itemize}

\section{Handling Fault Tolerance and Resilience}
This section addresses system reliability:
\begin{itemize}
    \item Fault tolerance mechanisms:
        \begin{itemize}
            \item Node failure handling
            \item Partition recovery
            \item Data availability maintenance
        \end{itemize}
    \item Resilience strategies
    \item System recovery procedures
\end{itemize}

\subsection{Data Sources}
\begin{itemize}
    \item Industrial sensors and devices
    \item Data stream generators
    \item Network interfaces
\end{itemize}

\subsection{Processing Nodes}
\begin{itemize}
    \item Autonomous decision-making units
    \item Local state management
    \item Communication protocols
\end{itemize}

\subsection{Communication Infrastructure}
\begin{itemize}
    \item Message passing system
    \item Network topology management
    \item Fault tolerance mechanisms
\end{itemize}

\subsection{Node Implementation}
\begin{lstlisting}[language=Python]
class AutonomousNode:
    def __init__(self, node_id, neighbors):
        self.id = node_id
        self.neighbors = neighbors
        self.partition = None
        self.state = {}
    
    def compute_metric(self):
        # Implementation of metric computation
        pass
    
    def make_decision(self):
        # Implementation of decision making
        pass
    
    def communicate(self):
        # Implementation of communication
        pass
\end{lstlisting}

\subsection{Communication Protocols}
\begin{itemize}
    \item Message formats and serialization
    \item Protocol state machines
    \item Error handling and recovery
\end{itemize}

\subsection{Task Graph Representation}
\begin{lstlisting}[language=Python]
class TaskGraph:
    def __init__(self):
        self.tasks = {}
        self.dependencies = {}
    
    def add_task(self, task_id, requirements):
        # Implementation of task addition
        pass
    
    def add_dependency(self, from_task, to_task):
        # Implementation of dependency addition
        pass
\end{lstlisting}

\subsection{Execution Engine}
\begin{itemize}
    \item Task scheduling
    \item Resource allocation
    \item Load balancing
\end{itemize}

\subsection{Node Failure Detection}
\begin{lstlisting}[language=Python,caption=Node Failure Detection]
# Periodically send heartbeat messages
def send_heartbeat():
    while True:
        send_heartbeat_message()
        time.sleep(HEARTBEAT_INTERVAL)

# Monitor response times
def monitor_responses():
    while True:
        response_time = measure_response_time()
        if response_time > TIMEOUT_THRESHOLD:
            mark_node_as_failed()
            trigger_recovery_procedures()
\end{lstlisting}

\subsection{Data Replication}
\begin{itemize}
    \item Replication strategies
    \item Consistency protocols
    \item Recovery procedures
\end{itemize}

\subsection{Partition Recovery}
\begin{lstlisting}[language=Python,caption=Partition Recovery]
# Detect partition failure
def detect_partition_failure():
    if not receive_heartbeat():
        return True
    return False

# Identify affected nodes
def identify_affected_nodes():
    return get_nodes_in_partition()

# Reassign nodes to new partitions
def reassign_nodes():
    nodes = identify_affected_nodes()
    for node in nodes:
        assign_to_new_partition(node)

# Restore data consistency
def restore_consistency():
    synchronize_data()
    verify_consistency()
\end{lstlisting} 